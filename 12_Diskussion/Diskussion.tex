\chapter{Diskussion}
Igennem projektet, er det blevet klart for gruppens medlemmer at der er utallige måder at udvikle en web-applikation på. Grundet det minimale kendskab til webudviklingen i starten af projektet, er udviklingsforløbet, beslutninger og produktet blevet præget meget heraf. Alle på nær 1 person havde ingen kendskab til webudvikling i starten af projektet, hvormed det har været i fokus, at alle har skulle lære så meget som muligt.\\
Hver person har stået for, eller været med til hele udviklingsprocessen, lige fra database, til front end og design. Dette ville typisk være sådan, at hver udvikler specialiserer sig inden for fx. front-end eller back-end udvikling, hvormed der måske kunne være udviklet en bedre og mere omfattende web-applikation. ASP.NET Core MVC har dog vist sig til at være et rigtigt godt udviklings redskab til at lære at udvikle web-applikationer, fordi det har mange indbyggede funktionaliteter, såsom razor pages, gøres det nemt og hurtigt at opbygge en webapplikation dynamisk. Samtidigt er der stadig mulighed for at anvende Javascript, samt lave Actions i controllers, der returnerer Json. !Der er altså mulighed for en lang række af ting, dog kan det virke lidt forvirrende, at finde ud af, hvilke Actions der returnerer hvad. Sagt på en anden måde: implementeringen er ikke særligt konsekvent eller effektiv. I stedet for at sende HTML, som fylder væsentligt mere, kan man med fordel sende data, som bruges til at opdatere det eksisterende HTML på siden. ! \store{Skal nok omformulereres lidt (rød tråd) - fra start ! til slut !} \\

\noindent Valget af US's over Use Cases har gjort udviklingen væsentligt mere fri. Der er fra start af ikke blevet fastlagt en meget specifik struktur over, hvordan forskellige funktionaliteter skal implementeres, men blot hvad der skal implementeres. Det har givet den individuelle udvikler frihed til at bestemme, hvordan bla. design og implementering foretages. Dette har været en kæmpe fordel, idet beslutninger taget i UC's viser sig ikke at kunne implementeres, eller ikke giver mening. Samtidigt bliver det muligt for den enkelte udvikler at lave sit design af funktionaliteten om, uden at skulle tilbage og ændre UC. Dette er dog ikke et problem med US's, da disse er mere fleksible. US's har altså haft en positiv indflydelse på den agile software udvikling. \\

\noindent At have ansvaret for en enkelt widget, og ikke en overordnet del af projektet, har også medført at udviklingen har kunnet foregå i hver enkelt persons tempo. Man er således ikke afhængig af, at en anden udvikler skal være færdig for, at det er muligt at teste det man selv laver. Samtidigt har den lave kobling, som et MVC projekt medfører, gjort dette muligt, idet systemets forskellige widgets er uafhængige af hinanden. Dog er systemet afhængige af grupper og brugere fungerer, da det er fundamentet for, at alt andet virker, hvormed dette også prioriteres i starten af udviklingsforløbet.

\store{Nogle af tingene passer mere til analyse, og vi burde måske diskutere resultaterne af accpettest og diskurtere, problematikker med henblik på problemformuleringen. (Dette laves fælles.)}