\documentclass[11pt,Danish,a4paper,oneside,openright,final]{memoir} 
%% Memoir are using an emulation of the ccaption package
%% so to use the caption package without warnings we first
%% DisEmulate the ccaption package after it have been loaded
%% and then call the package we want to use.
%% For more information, check the documentation
%% https://www.ctan.org/pkg/memoir?lang=en
\RequireAtEndPackage{ccaption}{\DisemulatePackage{ccaption}}
\RequireAtEndPackage{ccaption}{\usepackage{caption}}
\usepackage[danish]{babel}
\usepackage[T1]{fontenc}			% inkluder dansk i overskrifter
\usepackage[utf8]{inputenc}	% inkluder dansk i overskrifter
\usepackage[table,xcdraw,pdftex,dvipsnames]{xcolor}
\usepackage[hidelinks=true]{hyperref}

\setlength{\parindent}{0pt}
\nonzeroparskip

%% Memoir commands for changeing whitespace before and after 
%% sec, subsec, subsubsec, para or subpara. Replace "S" with 
%% the thing you want to change the whitespace for 
%%(see documentation section 6.6)
%\setbeforeSskip{ <skip> }
%\setafterSskip{ <skip> }

\usepackage{xargs}                      % Use more than one optional parameter in a new commands
\usepackage{booktabs}
\usepackage[colorinlistoftodos,prependcaption,textsize=tiny]{todonotes}
\newcommandx{\store}[2][1=]{\todo[linecolor=red,backgroundcolor=red!25,bordercolor=red,#1]{#2}}
\newcommandx{\jonathan}[2][1=]{\todo[linecolor=blue,backgroundcolor=blue!25,bordercolor=blue,#1]{#2}}
\newcommandx{\thomas}[2][1=]{\todo[linecolor=green,backgroundcolor=green!25,bordercolor=green,#1]{#2}}
\newcommandx{\tobycat}[2][1=]{\todo[linecolor=cyan,backgroundcolor=cyan!25,bordercolor=cyan,#1]{#2}}
\newcommandx{\mini}[2][1=]{\todo[linecolor=violet,backgroundcolor=violet!25,bordercolor=violet,#1]{#2}}
\newcommandx{\florent}[2][1=]{\todo[linecolor=magenta,backgroundcolor=magenta!25,bordercolor=magenta,#1]{#2}}
\newcommandx{\valdemar}[2][1=]{\todo[linecolor=orange,backgroundcolor=orange!25,bordercolor=orange,#1]{#2}}


\usepackage{listings}
\usepackage{url}
\usepackage{lmodern}
\usepackage{varioref} %% \vref gives you references including pages
\usepackage[fleqn]{amsmath} %
\usepackage[fleqn]{mathtools}				% Andre matematik- og tegnudvidelser
\usepackage[version=3]{mhchem} 				% Kemi-pakke til flot og let notation af formler, f.eks. \ce{Fe2O3}
\usepackage{siunitx}						% Flot og konsistent praesentation af tal og enheder med \si{enhed} og \SI{tal}{enhed}
%\sisetup{locale=DE}							% Opsaetning af \SI (DE for komma som decimalseparator)
\usepackage[utf8]{inputenc}	
\usepackage{pdfpages}			% G�r det muligt at inkludere pdf-dokumenter med kommandoen \includepdf[pages=]{fil.pdf}	
\usepackage{multicol}
\usepackage{multirow}
\usepackage{color} %% Colored text
\usepackage{amsfonts}
\usepackage{amssymb}
\usepackage{amsopn}
\usepackage{latexsym}
\usepackage{amstext}
\usepackage{longtable} %% tables that spans multiple pages
\usepackage{mathrsfs} %% nice math text/symbols

\usepackage{color}
 
\definecolor{dkgreen}{rgb}{0,0.6,0}
\definecolor{gray}{rgb}{0.5,0.5,0.5}
\definecolor{mauve}{rgb}{0.58,0,0.82}
\definecolor{lightgray}{rgb}{0.95,0.95,0.95}

% When including these 3 lines, it now enumerates with C.S.L where C is chapter number, S is section number and L is list number
\usepackage{enumitem}
\setenumerate[1]{label=\thesection.\arabic*}
\setenumerate[2]{label*=\arabic*}
% It is now also possible to continue numbering by using \begin{enumerate}[resume]

\makeatletter
\renewcommand*\env@matrix[1][*\c@MaxMatrixCols c]{%
  \hskip -\arraycolsep
  \let\@ifnextchar\new@ifnextchar
  \array{#1}}
\makeatother

%
\setheaderspaces{*}{5\onelineskip}{*}
\makepagestyle{sitin}

% Margin
\setlrmarginsandblock{*}{3.5cm}{0.75} % højre og venstre
\setulmarginsandblock{3cm}{*}{0.75}    % top og bund
\checkandfixthelayout[nearest]        % specifikt valg af højde algoritme
\renewcommand{\marginparwidth}{75pt}

\makeoddhead{sitin}
	%Left
	{
	    Gruppe 12
	}
	%Center
	{
		\small\rightmark
	}
	%Right
	{
	    \includegraphics[height=4\onelineskip]{03_Preamble/LargeAULogo.png}
	}

\makeevenhead{sitin}
	%Left
	{
		\includegraphics[height=4\onelineskip]{LargeAULogo.png}
	}
	%Center
	{
		\small\leftmark 
	}
	%Right
	{
	    Gruppe 12
	}
	
\makeoddfoot{sitin}{}{}{\thepage}
\makeevenfoot{sitin}{}{}{\thepage}

\makeheadrule{sitin}{\textwidth}{.4pt}
\makefootrule{sitin}{\textwidth}{.4pt}{0.1cm}

\pagestyle{sitin}



%\usepackage[dvipsnames]{xcolor}
%\newcommand\tab[1][1cm]{\hspace*{#1}}
\usepackage[T1]{fontenc}
\usepackage{afterpage}
\usepackage{siunitx}
\usepackage{transparent}
\usepackage{hyperref}
%\usepackage[backend=bibtex,style=numeric,bibencoding=ascii
%style=alphabetic
%style=reading]{biblatex}
%\usepackage[
%singlelinecheck=false % <-- important
%]{caption}

% Change chapter pages
\copypagestyle{chapter}{plain}
\makeoddfoot{chapter}{}{}{\small\thepage}
\makeevenfoot{chapter}{\small\thepage}{}{}
\makefootrule{chapter}{\textwidth}{\normalrulethickness}{\footruleskip}


%
% Section titles
%
\settocdepth{paragraph}
\setsecnumdepth{paragraph}
\maxsecnumdepth{paragraph}
\setsecheadstyle{\Large\bfseries\sffamily\raggedright}
\setsubsecheadstyle{\large\bfseries\sffamily\raggedright}
\setsubsubsecheadstyle{\normalsize\bfseries\sffamily\raggedright}
\raggedbottomsectiontrue

%
% Table of Contents
%
\renewcommand{\contentsname}{Table of Contents}
\makeatletter
\setlength{\cftpartnumwidth}{2em}% Set length of number width in ToC for \part
\setlength{\cftchapternumwidth}{2em}% Set length of number width in ToC for \chapter
\setlength{\cftsectionnumwidth}{3em}% Set length of number width in ToC for \section
\setlength{\cftsubsectionnumwidth}{4em}% Set length of number width in ToC for \subsection
\setlength{\cftsubsubsectionnumwidth}{5em}% Set length of number width in ToC for \subsection
%\setlength{\cftparagrnumwidth}{5em}% Set length of number width in ToC for \subsection
\makeatother


\usepackage{calc}

% Define a new chapter style
\makeatletter
\makechapterstyle{worksheet}{
	%% Memoir commands for changeing whitespace before and after 
	%% chapter. Note that the outcommented values are the defaults
	%% \setlength{\beforechapskip}{50pt}
	%% \setlength{\afterchapskip}{40pt}
	\setlength{\beforechapskip}{1ex}
	\setlength{\midchapskip}{0pt}
	\setlength{\afterchapskip}{1ex}
 	\newcommand{\chapterrule}{\rule[.2\baselineskip]{\textwidth}{1pt}}
  	\renewcommand\chapnamefont{\Large\sffamily}
  	\renewcommand\chapnumfont{\Large\sffamily\centering}
  	\renewcommand\chaptitlefont{\huge\bfseries\sffamily\centering}
  	\renewcommand\printchaptertitle[1]{%
    \chaptitlefont
    	\ifdim\@tempdimc > 0pt\relax% one line
      		\chapterrule \\
      		##1
      		\chapterrule
    	\else% two+ lines
        	>{\chaptitlefont\arraybackslash}p{\textwidth-2\tabcolsep}
     		\chapterrule \\
      		\phantomsection
      		\addtocontents{toc}{\protect\contentsline{chapter}{\protect\numberline{}##1}{}{chapter*.\thepage}}
      		##1
      		\chapterrule
    	\fi
	}
}
\makeatother
\chapterstyle{worksheet}

\usepackage{acronym}
\usepackage{nicefrac}
\usepackage{placeins}
\usepackage{graphicx}
\usepackage{subcaption}
\usepackage{rotating}
\usepackage{float}
\usepackage{epstopdf} % allows to use eps files in figures
\epstopdfsetup{outdir=./epstopdf/}
\newcommand{\matt}[1]{\bar{\mathbf{#1}}} % Laver Matrix notation med dobbelt overline og fed skrift
%% Reference to different figs and tables
\newcommand{\figref}[1]{Fig.~\ref{#1}}
\newcommand{\secref}[1]{Section~\ref{#1}}
\newcommand{\chapref}[1]{Chapter~\ref{#1}}
\newcommand{\appref}[1]{Appendix~\ref{#1}}
\newcommand{\tabref}[1]{Table~\ref{#1}}
\newcommand{\listref}[1]{Listing~\ref{#1}}
\newcommand{\eref}[1]{(\ref{#1})}

\newcommand\myworries[1]{\textcolor{red}{#1}}

%% Math Commands
\newcommand{\abs}[1]{\ensuremath{\left\vert #1\right\vert}}
\newcommand{\norm}[1]{\ensuremath{\left\vert\left\vert #1\right\vert\right\vert}}
\newcommand{\mtrix}[1]{\ensuremath{\boldsymbol{\ensuremath{\underline{#1}}}}}
\newcommand{\vektor}[1]{\ensuremath{\boldsymbol{\ensuremath{{#1}}}}}
\newcommand{\qaxis}[1]{\ensuremath{\boldsymbol{\ensuremath{\hat{#1}}}}}
\newcommand{\nicefag}[2]{\left[\nicefrac{#1}{#2}\right]}
\newcommand{\fag}[2]{\left[\frac{#1}{#2}\right]}
\newcommand{\rank}[1]{\ensuremath{\operatorname{rank\left(#1\right)}}}
%% Custom commands
\newcommand{\rott}[1]{\begin{sideways}#1\end{sideways}}
\newcommand{\tabitem}{~~\llap{\textbullet}~~}

% The framed package is used in the example environment
\usepackage{framed}
% Show the frame of the page segments for placements
%\usepackage{showframe}
% Count chapters, makes it possible to autoupdate number of appendices
\usepackage{totcount}

% To make different kinds of diagrams
\usepackage{tikz}
\usetikzlibrary{calc}
\usepackage{schemabloc} % Documentation only in French 
						% https://www.ctan.org/pkg/schemabloc
\usepackage{blox}		% Does the same as  the schemabloc package, but
						% documentation is in English 
						%https://www.ctan.org/pkg/blox
\usetikzlibrary{circuits}
\usetikzlibrary{positioning}

%%%%%%%%%%%%%%%%%%%%%%%%%%%%%%%%%%%%%%%%%%%
%% IF YOU NEED A PACKAGE, PUT UNDER THIS %%
%%%%%%%%%%%%%%%%%%%%%%%%%%%%%%%%%%%%%%%%%%%

\usepackage{tabto}
\usepackage{textcomp}
\usepackage{upgreek}
\usepackage{mathptmx}
\usepackage{array}
%\usepackage[
 % disable, %turn off todonotes
  %colorinlistoftodos, %enable a coloured square in the list of %todos
  %textwidth=\marginparwidth, %set the width of the todonotes
  %textsize=scriptsize, %size of the text in the todonotes
  %]{todonotes}
\usepackage[colorinlistoftodos]{todonotes}

\usepackage{circuitikz}%enables electriacls curcits in tikz
%\usepackage{slashbox} % So two different things can be written in a box i an table
\usepackage{diagbox} % modern version of slashbox
\usepackage{cancel} %the ability to strikeout in equations

\usepackage{arydshln} % Dashed hline within table enviroment

% Used to make flowchart (ran out of blocks in lucidchart)
\usetikzlibrary{shapes.geometric, arrows}

%%%%%%%%%%%%%%%%%%%%%%%%%%%%%%%%%%%%%%%%%%%
%% IF YOU NEED A PACKAGE, PUT ABOVE THIS %%
%%%%%%%%%%%%%%%%%%%%%%%%%%%%%%%%%%%%%%%%%%%


%%%%%%%%%%%%%%%%%%%%%%%%%%%%%%%%%%%%%%%%%%%%%%%%
% Bibliography
% http://en.wikibooks.org/wiki/LaTeX/Bibliography_Management
%%%%%%%%%%%%%%%%%%%%%%%%%%%%%%%%%%%%%%%%%%%%%%%%
\usepackage[square,numbers]{natbib}
% Add the \citep{key} command which display a
% reference as [author, year]
%\usepackage[authoryear]{natbib}
%\setcitestyle{round,nonamebreak}
%\bibpunct{(}{)}{;}{a}{}{,}
% Appearance of the bibliography
\bibliographystyle{IEEEtran}


%%%%%%%%%%%%%%%%%%%%%%%%%%%%%%%%%
%% THIS MUST BE THE LAST THING %%
%%%%%%%%%%%%%%%%%%%%%%%%%%%%%%%%%
% \usepackage[hidelinks,breaklinks]{hyperref}
% \hypersetup{
% 	pdfpagelabels=true,%
% 	plainpages=false,%
% 	pdfauthor={gruppe 12,
%                 3. semester,
%                 AU,
%                 Aarhus,
%                 Denmark},%
% 	pdftitle={Drink master},%
% 	pdfsubject={Drink master},%
% 	bookmarksnumbered=true,%
% 	colorlinks=false,%
% 	pdfstartview=FitH,%
% 	pdfduplex=DuplexFlipLongEdge,
% 	pdfkeywords={gruppe 12,
%                 Aarhus,
%                 AU,
%               },
% 	breaklinks
% }

\usepackage{minted}
\definecolor{LightGray}{gray}{0.9}
\definecolor{bluekeywords}{rgb}{0,0,1}
\definecolor{greencomments}{rgb}{0,0.5,0}
\definecolor{redstrings}{rgb}{0.64,0.08,0.08}
\definecolor{xmlcomments}{rgb}{0.5,0.5,0.5}
\definecolor{types}{rgb}{0.17,0.57,0.68}

\usepackage{listings}
\lstset{language=C++,
captionpos=b,
%numbers=left, %Nummerierung
%numberstyle=\tiny, % kleine Zeilennummern
frame=lines, % Oberhalb und unterhalb des Listings ist eine Linie
showspaces=false,
showtabs=false,
breaklines=true,
showstringspaces=false,
breakatwhitespace=true,
escapeinside={(*@}{@*)},
commentstyle=\color{greencomments},
morekeywords={char, void, var, value, get, set,DrinkMachineClient},
keywordstyle=\color{bluekeywords},
stringstyle=\color{redstrings},
basicstyle=\ttfamily\small,
}

\lstdefinelanguage{Gherkin}{
  keywords={Naar, Saa, Givet, Og},
  ndkeywords={Egenskab, Baggrund, Scenarie},
  sensitive=false,
  comment=[l]{\#},
  morestring=[b]',
  morestring=[b]"
}

\usepackage{memhfixc} %% Include this package after hyperref when using memoir