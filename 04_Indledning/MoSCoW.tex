For at begrænse projektets indhold til en passende størrelse, foretages en afgrænsning af projektet. Dette gør det muligt at fastlægge nogle krav, som der kan tages udgangspunkt i udviklingsprocessen. Disse kan da prioriteres ud fra forskellige parametre som tid, forhåndskendskab og vigtighed, hvorved udviklingsprocessen kan struktureres efter denne prioritering. For at lave denne afgrænsning, foretages en MoSCoW analyse, som kan ses i afsnit \ref{sec:MoSCoW}.

\subsection{MoSCoW} \label{sec:MoSCoW}
I starten af udviklingsforløbet foretages en MoSCoW analyse, for at få et overblik over applikationens ønskede funktionalitet, samt for at prioritere implementeringen af applikationens elementer. Resultatet af denne analyse kan ses herunder.

\noindent \textbf{Must have:}
\begin{itemize}
    \item Applikationen skal have en login-funktion og mulighed for at oprette brugere
    \item En bruger skal identificeres ved e-mail som anvendes ved login
    \item I applikationen skal det være muligt at oprette en gruppe med et default layout, hvortil der kan tilføjes medlemmer
    \item Administrator rettigheder for grupper
    \item Applikationen skal indeholde en kalender
    \item Applikationen skal indeholde funktionalitet der gør det muligt at se og planlægge en madplan og rengøringsplan.
    \item Det skal være muligt at bytte vagter mellem brugere. 
    \item Gruppens aktiviteter skal vises på gruppens kalender.
    \item Applikationen skal indeholde en opslagstavle for en gruppe.
    \item Der skal kunne oprettes forbindelse til serveren ved brug af HTTP.
    \item Applikationen skal indeholde en liste widget.
    \item Det skal være muligt for gruppens medlemmer at tilføje widgets på gruppens side.
    \item Det skal være muligt for en gruppes administrator at fjerne/redigere i widgets på gruppens side.
    \item Applikationen skal gøre det muligt at oprette et booking system til reservation af ressourcer.
    \item Applikationen skal have en regnskabs widget.
\end{itemize}

\noindent \textbf{Should have:}
\begin{itemize}
    \item Brugere skal kunne modtage notifikationer om ændringer i widgets.
    \item Mulighed for at tilpasse udseendet af gruppen.
    \item Owner rettigheder for grupper.
    \item Projektets applikation hostes på dens eget domæne.
\end{itemize}

\noindent \textbf{Could have:}
\begin{itemize}
    \item Mulighed for at indsamle brugerstatistikker.
    \item Flere loginmuligheder via social media-accounts.
    \item Mulighed for at vælge andre sprog.
\end{itemize}

\noindent \textbf{Won't have (this time)}
\begin{itemize}
    \item En tilsvarende App til iOS/Android.
    \item Integrering med mobilePay/ weshare mm.
    \item Integrering med dansk supermarked der udbyder levering af varer.
    \item Søgemaskine optimering.
\end{itemize}

