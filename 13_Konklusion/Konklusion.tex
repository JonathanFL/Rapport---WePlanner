\chapter{Konklusion}

I bofællesskaber kan det være svært at inddrage alle beboernes dagligdag, hvor strukturen i hjemmet kan gå tabt og hverdagen kan blive hektisk for alle beboerne. Denne webapplikation er et redskab for bofællesskaber, som kan strukturere og organisere dagligdagen for bofællesskabet, men samtidig give hver person i hjemmet de selvsamme fordele og ro i hverdagen. Applikationen søger at løse problemet hvad verbal kommunikation ikke kan gøre ved et bofællesskab, nemlig koordering af alle beboers dagligdag. \store{kan bruges i indledning, men bør ikke beskrives i konklusionen. }

Det er lykkedes for gruppen at have et funktionsdygtigt produkt, hvor alle Must Have krav er blevet opfyldt, samt de fleste af Should have.

Produktet har været en succes, hvor der er få enkelte mangler og steder, hvor det kunne forbedres. \store{skal uddybes}

På hjemmesiden ville man kunne se at man kan oprette sig som bruger, for at dernæst kunne oprette grupper bestående af flere brugere. For grupperne på hjemmesiden, vil det kunne ses at at der kan oprettes forskellige widgets for gruppen, med hver deres funktionalitet. Alle widgets vil blive samlet på gruppens dashboard, som er gruppens central punkt. 
Med disse widgets vil man så kunne strukturere på gruppens dagligdag. Der er mulighed for at lave opslag wall, der kan laves planer for gruppen i Planning widget. På booking widget vil der kunne blive oprettet ressourcer, som eksempelvis en vaskemaskine, og vil kunne blive booket, hvor der kan ses en oversigt over hvem der skal bruge den og hvornår. Alle begivenheder inklusive bookinger, for en gruppe kan ses på gruppekalenderen, hvor der også kan laves/slettes begivenheder.  gruppen hvor b   hvor man har mulighed for at lave opslag, booke ressourcer og betalinger for gruppen blandt andre ting. 

Gruppens store ønske var at implementere applikationen, så det også kunne bruges på flere platforme, samt en app. Størrelsen af punktet, gjorde at det fra begyndelsen blev nedprioriteret og blev sat som et Won't have krav. 

Webapplikationen er heller ikke blevet udarbejdet således, at der kan benyttes kontier fra andre medier, eksempelvis Facebook, hvilket jo er meget anvendeligt for en applikation som denne. 

\store{Den skal forbedres og præciseres bedre. Uddyb måske MosCow, og konkludere på punkterne. Væsentlige kvantitative resultater kan nævnes, hvorimod den detaljerede redegørelse og diskussion henvises til Diskussion-afsnittet. Processen skal også konkluderes, hvilket skal gøres til sidst i konklusionen.}

\section{Perspektivering}