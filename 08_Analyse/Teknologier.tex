\section{Valg af teknologier}
\valdemar{skal omskrives}
Da kendskabet til webudvikling ikke var så udbredt, blev der undersøgt forskellige muligheder for, hvordan projektet skulle udvikles. Der blev i begyndelsen undersøgt teknologierne \textit{ASP.NET Web Forms}, \textit{ASP.NET MVC}. Der blev undersøgt hvor nemme de var at sætte sig ind i, samtidigt med at det skal kunne understøtte den nødvendige funktionalitet. Der var en del fordele ved ASP.NET MVC, hvor man bl.a. kunne strukturere alt i Models, Views og Controllers, som gav god separation of concern. Desuden har ASP.NET MVC en html type, cshtml, der tager brug af Razor, som er et mix af HTML og C\# kode. Dette giver en masse muligheder for at ændre view i forhold til den Model der er modtaget fra controlleren. \valdemar{Henvisning til bilag}
%I databasekurset I4DAB, blev der til start undervist i Entity Framework Core, hvilket mere eller mindre tvang projektet til også at bruge ASP.NET MVC CORE, for at have kompatibilitet med det samme framework der blev undervist i.
\valdemar{...}

\noindent Valget endte på ASP.NET MVC, hvortil Identity Frameworket, der giver database opsætning for login, med hashing af passwords er blevet tilføjet. 